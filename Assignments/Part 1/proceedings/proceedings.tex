\documentclass{sigchi}

% Use this section to set the ACM copyright statement (e.g. for
% preprints).  Consult the conference website for the camera-ready
% copyright statement.

% Copyright
\CopyrightYear{2021}
%\setcopyright{acmcopyright}
\setcopyright{acmlicensed}
%\setcopyright{rightsretained}
%\setcopyright{usgov}
%\setcopyright{usgovmixed}
%\setcopyright{cagov}
%\setcopyright{cagovmixed}
% DOI
\doi{https://doi.org/10.1145/3313831.XXXXXXX}
% ISBN
\isbn{978-1-4503-6708-0/20/04}
%Conference
\conferenceinfo{CHI'20,}{April  25--30, 2020, Honolulu, HI, USA}
%Price
\acmPrice{\$15.00}

% Use this command to override the default ACM copyright statement
% (e.g. for preprints).  Consult the conference website for the
% camera-ready copyright statement.

%% HOW TO OVERRIDE THE DEFAULT COPYRIGHT STRIP --
%% Please note you need to make sure the copy for your specific
%% license is used here!
% \toappear{
% Permission to make digital or hard copies of all or part of this work
% for personal or classroom use is granted without fee provided that
% copies are not made or distributed for profit or commercial advantage
% and that copies bear this notice and the full citation on the first
% page. Copyrights for components of this work owned by others than ACM
% must be honored. Abstracting with credit is permitted. To copy
% otherwise, or republish, to post on servers or to redistribute to
% lists, requires prior specific permission and/or a fee. Request
% permissions from \href{mailto:Permissions@acm.org}{Permissions@acm.org}. \\
% \emph{CHI '16},  May 07--12, 2016, San Jose, CA, USA \\
% ACM xxx-x-xxxx-xxxx-x/xx/xx\ldots \$15.00 \\
% DOI: \url{http://dx.doi.org/xx.xxxx/xxxxxxx.xxxxxxx}
% }

% Arabic page numbers for submission.  Remove this line to eliminate
% page numbers for the camera ready copy
% \pagenumbering{arabic}

% Load basic packages
\usepackage{balance}       % to better equalize the last page
\usepackage{graphics}      % for EPS, load graphicx instead 
\usepackage[T1]{fontenc}   % for umlauts and other diaeresis
\usepackage{txfonts}
\usepackage{mathptmx}
\usepackage[pdflang={en-US},pdftex]{hyperref}
\usepackage{color}
\usepackage{booktabs}
\usepackage{textcomp}

% Some optional stuff you might like/need.
\usepackage{microtype}        % Improved Tracking and Kerning
% \usepackage[all]{hypcap}    % Fixes bug in hyperref caption linking
\usepackage{ccicons}          % Cite your images correctly!
% \usepackage[utf8]{inputenc} % for a UTF8 editor only

% If you want to use todo notes, marginpars etc. during creation of
% your draft document, you have to enable the "chi_draft" option for
% the document class. To do this, change the very first line to:
% "\documentclass[chi_draft]{sigchi}". You can then place todo notes
% by using the "\todo{...}"  command. Make sure to disable the draft
% option again before submitting your final document.
\usepackage{todonotes}

% Paper metadata (use plain text, for PDF inclusion and later
% re-using, if desired).  Use \emtpyauthor when submitting for review
% so you remain anonymous.
\def\plaintitle{Preventing Bias in Machine Learning by using Bias Aware: An Empirical Experimental Study}
\def\plainauthor{First Author, Second Author, Third Author,
  Fourth Author, Fifth Author, Sixth Author}
\def\emptyauthor{}
\def\plainkeywords{Authors' choice; of terms; separated; by
  semicolons; include commas, within terms only; this section is required.}
\def\plaingeneralterms{Documentation, Standardization}

% llt: Define a global style for URLs, rather that the default one
\makeatletter
\def\url@leostyle{%
  \@ifundefined{selectfont}{
    \def\UrlFont{\sf}
  }{
    \def\UrlFont{\small\bf\ttfamily}
  }}
\makeatother
\urlstyle{leo}

% To make various LaTeX processors do the right thing with page size.
\def\pprw{8.5in}
\def\pprh{11in}
\special{papersize=\pprw,\pprh}
\setlength{\paperwidth}{\pprw}
\setlength{\paperheight}{\pprh}
\setlength{\pdfpagewidth}{\pprw}
\setlength{\pdfpageheight}{\pprh}

% Make sure hyperref comes last of your loaded packages, to give it a
% fighting chance of not being over-written, since its job is to
% redefine many LaTeX commands.
\definecolor{linkColor}{RGB}{6,125,233}
\hypersetup{%
  pdftitle={\plaintitle},
% Use \plainauthor for final version.
%  pdfauthor={\plainauthor},
  pdfauthor={\emptyauthor},
  pdfkeywords={\plainkeywords},
  pdfdisplaydoctitle=true, % For Accessibility
  bookmarksnumbered,
  pdfstartview={FitH},
  colorlinks,
  citecolor=black,
  filecolor=black,
  linkcolor=black,
  urlcolor=linkColor,
  breaklinks=true,
  hypertexnames=false
}

% create a shortcut to typeset table headings
% \newcommand\tabhead[1]{\small\textbf{#1}}

% End of preamble. Here it comes the document.
\begin{document}

\title{\plaintitle}

\numberofauthors{1}
\author{%
  \alignauthor{Andy Gray\\
    \affaddr{445348}\\
    \email{445348@swansea.ac.uk}}\\
}

\maketitle

\begin{abstract}
 
\end{abstract}


% ACM Classfication

\begin{CCSXML}
<ccs2012>
<concept>
<concept_id>10003120.10003121</concept_id>
<concept_desc>Human-centered computing~Human computer interaction (HCI)</concept_desc>
<concept_significance>500</concept_significance>
</concept>
<concept>
<concept_id>10003120.10003121.10003125.10011752</concept_id>
<concept_desc>Human-centered computing~Haptic devices</concept_desc>
<concept_significance>300</concept_significance>
</concept>
<concept>
<concept_id>10003120.10003121.10003122.10003334</concept_id>
<concept_desc>Human-centered computing~User studies</concept_desc>
<concept_significance>100</concept_significance>
</concept>
</ccs2012>
\end{CCSXML}

\ccsdesc[500]{Human-centered computing~Human computer interaction (HCI)}
\ccsdesc[300]{Human-centered computing~Haptic devices}
\ccsdesc[100]{Human-centered computing~User studies}

% Author Keywords
\keywords{\plainkeywords}

% Print the classficiation codes
\printccsdesc
Please use the 2012 Classifiers and see this link to embed them in the text: \url{https://dl.acm.org/ccs/ccs_flat.cfm}



\section{Introduction}

\subsection{Hypothesis or Conjecture}
To remove the potential gender bias in suggested pay to an employee from data that has a clear gender bias within the dataset. Using bias-aware algorithms to figure out how much data exists in the data to measure the bias then correctly and then remove the outcome's bais' effects.

\section{Main}

\subsection{Focus of the Study}
The study aims to remove bias within algorithms. This aim is within a context that there is an awareness of bias within the data. It is well documented and known that women are paid less than men for doing the same role. This situation is known as the gender pay gap. When companies are looking at how much to offer new workers or performance reviews to current employees, when current employees' data get used to forming how much a person should get paid, a man and women will receive different amounts. An ML model will likely figure this out. Therefore we aim to remove the bias and prejudice in the data to build a model representing the employees more fairly.

Additional libraries like Shapley Additive Exploration or possibly LIME will get used to gain some insights into the explainability of the models.
 

\subsection{Research Landscape and Social Significance Evidence}
ML requires many past data to inform future events, with AI and machine learning being the key driver behind many decisions. However, with there being a well-known gap between a person's gender and their pay, the ML models will only learn this and use this as a factor in their decisions making. Therefore, to stop this from happening, a system needs to be put into place to remove this process's bias. %add more facts here


\subsection{Outline the Experimental Method(s)}
The experiment will aim to plot the initial dataset to see where the decisions are for the different genders in questions. We will then aim to remove this gender bias by first identifying the bias and then removing it. 

We will also aim to use additional libraries to gain insights and explainability from the outputs to see how much of an impact the methods have had on removing gender bias.

\subsection{Data to be Used}
We will be using simulated data containing gender, years of experience and type of career initially. The overall aim is to predict the salary of someone while taking these features into account. We will also predict the salary of an employee while also removing any gender bias within the results. The type of career will be focusing on software engineering (SWE) and consulting.

As the initial dataset will be synthetic based on general assumptions about pay, which are well known, there will be a clear positive relationship between years of experience and a person's salary. An SWE will earn less than a consultant, and being male will earn them more money than females. Additional considerations within the data are that in SWE roles, women will start at the lower end of the scale while men will be varied and, therefore, women will be over time increasing their pay. However, this increase will be at a faster rate than men but from a lower starting point. While for consulting, both males and females will start at the same rate, but men will get more considerable increases in pay over time compared to their women counterparts.

\subsection{Concepts to be Discussed}

The study's concepts will be to look at removing gender bias from a predictive model and using tools to look at the explainability of the model and what gets used to create the predictions.

%\section{Conclusion}

%It is important that you write for the SIGCHI audience. Please read
%previous years' proceedings to understand the writing style and
%conventions that successful authors have used. It is particularly
%important that you state clearly what you have done, not merely what
%you plan to do, and explain how your work is different from previously
%published work, i.e., the unique contribution that your work makes to
%the field. Please consider what the reader will learn from your
%submission, and how they will find your work useful. If you write with
%these questions in mind, your work is more likely to be successful,
%both in being accepted into the conference, and in influencing the
%work of our field.

\section{Acknowledgments}

%Sample text: We thank all the volunteers, and all publications support
%and staff, who wrote and provided helpful comments on previous
%versions of this document. Authors 1, 2, and 3 gratefully acknowledge
%the grant from NSF (\#1234--2012--ABC). \textit{This whole paragraph is
%  just an example.}

% Balancing columns in a ref list is a bit of a pain because you
% either use a hack like flushend or balance, or manually insert
% a column break.  http://www.tex.ac.uk/cgi-bin/texfaq2html?label=balance
% multicols doesn't work because we're already in two-column mode,
% and flushend isn't awesome, so I choose balance.  See this
% for more info: http://cs.brown.edu/system/software/latex/doc/balance.pdf
%
% Note that in a perfect world balance wants to be in the first
% column of the last page.
%
% If balance doesn't work for you, you can remove that and
% hard-code a column break into the bbl file right before you
% submit:
%
% http://stackoverflow.com/questions/2149854/how-to-manually-equalize-columns-
% in-an-ieee-paper-if-using-bibtex
%
% Or, just remove \balance and give up on balancing the last page.
%
\balance{}


% BALANCE COLUMNS
\balance{}

% REFERENCES FORMAT
% References must be the same font size as other body text.
\bibliographystyle{SIGCHI-Reference-Format}
\bibliography{sample}

\end{document}

%%% Local Variables:
%%% mode: latex
%%% TeX-master: t
%%% End:
